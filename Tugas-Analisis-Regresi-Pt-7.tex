% Options for packages loaded elsewhere
\PassOptionsToPackage{unicode}{hyperref}
\PassOptionsToPackage{hyphens}{url}
%
\documentclass[
]{article}
\usepackage{amsmath,amssymb}
\usepackage{iftex}
\ifPDFTeX
  \usepackage[T1]{fontenc}
  \usepackage[utf8]{inputenc}
  \usepackage{textcomp} % provide euro and other symbols
\else % if luatex or xetex
  \usepackage{unicode-math} % this also loads fontspec
  \defaultfontfeatures{Scale=MatchLowercase}
  \defaultfontfeatures[\rmfamily]{Ligatures=TeX,Scale=1}
\fi
\usepackage{lmodern}
\ifPDFTeX\else
  % xetex/luatex font selection
\fi
% Use upquote if available, for straight quotes in verbatim environments
\IfFileExists{upquote.sty}{\usepackage{upquote}}{}
\IfFileExists{microtype.sty}{% use microtype if available
  \usepackage[]{microtype}
  \UseMicrotypeSet[protrusion]{basicmath} % disable protrusion for tt fonts
}{}
\makeatletter
\@ifundefined{KOMAClassName}{% if non-KOMA class
  \IfFileExists{parskip.sty}{%
    \usepackage{parskip}
  }{% else
    \setlength{\parindent}{0pt}
    \setlength{\parskip}{6pt plus 2pt minus 1pt}}
}{% if KOMA class
  \KOMAoptions{parskip=half}}
\makeatother
\usepackage{xcolor}
\usepackage[margin=1in]{geometry}
\usepackage{color}
\usepackage{fancyvrb}
\newcommand{\VerbBar}{|}
\newcommand{\VERB}{\Verb[commandchars=\\\{\}]}
\DefineVerbatimEnvironment{Highlighting}{Verbatim}{commandchars=\\\{\}}
% Add ',fontsize=\small' for more characters per line
\usepackage{framed}
\definecolor{shadecolor}{RGB}{248,248,248}
\newenvironment{Shaded}{\begin{snugshade}}{\end{snugshade}}
\newcommand{\AlertTok}[1]{\textcolor[rgb]{0.94,0.16,0.16}{#1}}
\newcommand{\AnnotationTok}[1]{\textcolor[rgb]{0.56,0.35,0.01}{\textbf{\textit{#1}}}}
\newcommand{\AttributeTok}[1]{\textcolor[rgb]{0.13,0.29,0.53}{#1}}
\newcommand{\BaseNTok}[1]{\textcolor[rgb]{0.00,0.00,0.81}{#1}}
\newcommand{\BuiltInTok}[1]{#1}
\newcommand{\CharTok}[1]{\textcolor[rgb]{0.31,0.60,0.02}{#1}}
\newcommand{\CommentTok}[1]{\textcolor[rgb]{0.56,0.35,0.01}{\textit{#1}}}
\newcommand{\CommentVarTok}[1]{\textcolor[rgb]{0.56,0.35,0.01}{\textbf{\textit{#1}}}}
\newcommand{\ConstantTok}[1]{\textcolor[rgb]{0.56,0.35,0.01}{#1}}
\newcommand{\ControlFlowTok}[1]{\textcolor[rgb]{0.13,0.29,0.53}{\textbf{#1}}}
\newcommand{\DataTypeTok}[1]{\textcolor[rgb]{0.13,0.29,0.53}{#1}}
\newcommand{\DecValTok}[1]{\textcolor[rgb]{0.00,0.00,0.81}{#1}}
\newcommand{\DocumentationTok}[1]{\textcolor[rgb]{0.56,0.35,0.01}{\textbf{\textit{#1}}}}
\newcommand{\ErrorTok}[1]{\textcolor[rgb]{0.64,0.00,0.00}{\textbf{#1}}}
\newcommand{\ExtensionTok}[1]{#1}
\newcommand{\FloatTok}[1]{\textcolor[rgb]{0.00,0.00,0.81}{#1}}
\newcommand{\FunctionTok}[1]{\textcolor[rgb]{0.13,0.29,0.53}{\textbf{#1}}}
\newcommand{\ImportTok}[1]{#1}
\newcommand{\InformationTok}[1]{\textcolor[rgb]{0.56,0.35,0.01}{\textbf{\textit{#1}}}}
\newcommand{\KeywordTok}[1]{\textcolor[rgb]{0.13,0.29,0.53}{\textbf{#1}}}
\newcommand{\NormalTok}[1]{#1}
\newcommand{\OperatorTok}[1]{\textcolor[rgb]{0.81,0.36,0.00}{\textbf{#1}}}
\newcommand{\OtherTok}[1]{\textcolor[rgb]{0.56,0.35,0.01}{#1}}
\newcommand{\PreprocessorTok}[1]{\textcolor[rgb]{0.56,0.35,0.01}{\textit{#1}}}
\newcommand{\RegionMarkerTok}[1]{#1}
\newcommand{\SpecialCharTok}[1]{\textcolor[rgb]{0.81,0.36,0.00}{\textbf{#1}}}
\newcommand{\SpecialStringTok}[1]{\textcolor[rgb]{0.31,0.60,0.02}{#1}}
\newcommand{\StringTok}[1]{\textcolor[rgb]{0.31,0.60,0.02}{#1}}
\newcommand{\VariableTok}[1]{\textcolor[rgb]{0.00,0.00,0.00}{#1}}
\newcommand{\VerbatimStringTok}[1]{\textcolor[rgb]{0.31,0.60,0.02}{#1}}
\newcommand{\WarningTok}[1]{\textcolor[rgb]{0.56,0.35,0.01}{\textbf{\textit{#1}}}}
\usepackage{graphicx}
\makeatletter
\def\maxwidth{\ifdim\Gin@nat@width>\linewidth\linewidth\else\Gin@nat@width\fi}
\def\maxheight{\ifdim\Gin@nat@height>\textheight\textheight\else\Gin@nat@height\fi}
\makeatother
% Scale images if necessary, so that they will not overflow the page
% margins by default, and it is still possible to overwrite the defaults
% using explicit options in \includegraphics[width, height, ...]{}
\setkeys{Gin}{width=\maxwidth,height=\maxheight,keepaspectratio}
% Set default figure placement to htbp
\makeatletter
\def\fps@figure{htbp}
\makeatother
\setlength{\emergencystretch}{3em} % prevent overfull lines
\providecommand{\tightlist}{%
  \setlength{\itemsep}{0pt}\setlength{\parskip}{0pt}}
\setcounter{secnumdepth}{-\maxdimen} % remove section numbering
\ifLuaTeX
  \usepackage{selnolig}  % disable illegal ligatures
\fi
\IfFileExists{bookmark.sty}{\usepackage{bookmark}}{\usepackage{hyperref}}
\IfFileExists{xurl.sty}{\usepackage{xurl}}{} % add URL line breaks if available
\urlstyle{same}
\hypersetup{
  pdftitle={Tugas Individu Analisis Regresi Pt 7},
  pdfauthor={Thufaillah Ulfah J},
  hidelinks,
  pdfcreator={LaTeX via pandoc}}

\title{Tugas Individu Analisis Regresi Pt 7}
\author{Thufaillah Ulfah J}
\date{2024-03-05}

\begin{document}
\maketitle

\hypertarget{library}{%
\section{Library}\label{library}}

\begin{Shaded}
\begin{Highlighting}[]
\FunctionTok{library}\NormalTok{(readxl)}
\FunctionTok{library}\NormalTok{(dplyr)}
\end{Highlighting}
\end{Shaded}

\begin{verbatim}
## 
## Attaching package: 'dplyr'
\end{verbatim}

\begin{verbatim}
## The following objects are masked from 'package:stats':
## 
##     filter, lag
\end{verbatim}

\begin{verbatim}
## The following objects are masked from 'package:base':
## 
##     intersect, setdiff, setequal, union
\end{verbatim}

\begin{Shaded}
\begin{Highlighting}[]
\FunctionTok{library}\NormalTok{(plotly)}
\end{Highlighting}
\end{Shaded}

\begin{verbatim}
## Loading required package: ggplot2
\end{verbatim}

\begin{verbatim}
## 
## Attaching package: 'plotly'
\end{verbatim}

\begin{verbatim}
## The following object is masked from 'package:ggplot2':
## 
##     last_plot
\end{verbatim}

\begin{verbatim}
## The following object is masked from 'package:stats':
## 
##     filter
\end{verbatim}

\begin{verbatim}
## The following object is masked from 'package:graphics':
## 
##     layout
\end{verbatim}

\begin{Shaded}
\begin{Highlighting}[]
\FunctionTok{library}\NormalTok{(lmtest)}
\end{Highlighting}
\end{Shaded}

\begin{verbatim}
## Warning: package 'lmtest' was built under R version 4.3.3
\end{verbatim}

\begin{verbatim}
## Loading required package: zoo
\end{verbatim}

\begin{verbatim}
## 
## Attaching package: 'zoo'
\end{verbatim}

\begin{verbatim}
## The following objects are masked from 'package:base':
## 
##     as.Date, as.Date.numeric
\end{verbatim}

\begin{Shaded}
\begin{Highlighting}[]
\FunctionTok{library}\NormalTok{(car)}
\end{Highlighting}
\end{Shaded}

\begin{verbatim}
## Loading required package: carData
\end{verbatim}

\begin{verbatim}
## 
## Attaching package: 'car'
\end{verbatim}

\begin{verbatim}
## The following object is masked from 'package:dplyr':
## 
##     recode
\end{verbatim}

\begin{Shaded}
\begin{Highlighting}[]
\FunctionTok{library}\NormalTok{(randtests)}
\FunctionTok{library}\NormalTok{(lmtest)}
\end{Highlighting}
\end{Shaded}

\hypertarget{data}{%
\section{Data}\label{data}}

\begin{Shaded}
\begin{Highlighting}[]
\NormalTok{data }\OtherTok{\textless{}{-}} \FunctionTok{read\_xlsx}\NormalTok{(}\StringTok{"C:}\SpecialCharTok{\textbackslash{}\textbackslash{}}\StringTok{Users}\SpecialCharTok{\textbackslash{}\textbackslash{}}\StringTok{MyBook Hype}\SpecialCharTok{\textbackslash{}\textbackslash{}}\StringTok{Downloads}\SpecialCharTok{\textbackslash{}\textbackslash{}}\StringTok{Data anreg pt7 .xlsx"}\NormalTok{)}
\NormalTok{data}
\end{Highlighting}
\end{Shaded}

\begin{verbatim}
## # A tibble: 15 x 2
##        X     Y
##    <dbl> <dbl>
##  1     2    54
##  2     5    50
##  3     7    45
##  4    10    37
##  5    14    35
##  6    19    25
##  7    26    20
##  8    31    16
##  9    34    18
## 10    38    13
## 11    45     8
## 12    52    11
## 13    53     8
## 14    60     4
## 15    65     6
\end{verbatim}

\hypertarget{model-regresi}{%
\section{Model Regresi}\label{model-regresi}}

\begin{Shaded}
\begin{Highlighting}[]
\NormalTok{model1 }\OtherTok{=} \FunctionTok{lm}\NormalTok{(}\AttributeTok{formula =}\NormalTok{ Y }\SpecialCharTok{\textasciitilde{}}\NormalTok{ X, data)}
\FunctionTok{summary}\NormalTok{(model1)}
\end{Highlighting}
\end{Shaded}

\begin{verbatim}
## 
## Call:
## lm(formula = Y ~ X, data = data)
## 
## Residuals:
##     Min      1Q  Median      3Q     Max 
## -7.1628 -4.7313 -0.9253  3.7386  9.0446 
## 
## Coefficients:
##             Estimate Std. Error t value Pr(>|t|)    
## (Intercept) 46.46041    2.76218   16.82 3.33e-10 ***
## X           -0.75251    0.07502  -10.03 1.74e-07 ***
## ---
## Signif. codes:  0 '***' 0.001 '**' 0.01 '*' 0.05 '.' 0.1 ' ' 1
## 
## Residual standard error: 5.891 on 13 degrees of freedom
## Multiple R-squared:  0.8856, Adjusted R-squared:  0.8768 
## F-statistic: 100.6 on 1 and 13 DF,  p-value: 1.736e-07
\end{verbatim}

\begin{Shaded}
\begin{Highlighting}[]
\NormalTok{model1}
\end{Highlighting}
\end{Shaded}

\begin{verbatim}
## 
## Call:
## lm(formula = Y ~ X, data = data)
## 
## Coefficients:
## (Intercept)            X  
##     46.4604      -0.7525
\end{verbatim}

Didapatkan Model Regresi sebagai berikut:
\[ \hat Y = 46.4604 - 0.7525X \] Model tersebut belum dikatakan sebagai
model terbaik karena belum melalui eksplorasi data dan serangkaian uji
asumsi serta normalitas untuk memperoleh model yang optimal.

\begin{Shaded}
\begin{Highlighting}[]
\FunctionTok{plot}\NormalTok{(}\AttributeTok{x=}\NormalTok{data}\SpecialCharTok{$}\NormalTok{X,}\AttributeTok{y=}\NormalTok{data}\SpecialCharTok{$}\NormalTok{Y)}
\end{Highlighting}
\end{Shaded}

\includegraphics{Tugas-Analisis-Regresi-Pt-7_files/figure-latex/unnamed-chunk-4-1.pdf}

Berdasarkan plot tersebut, dapat disimpulkan bahwa hubungan antara X dan
Y tidak linear.

\hypertarget{pemeriksaan-asumsi}{%
\section{Pemeriksaan Asumsi}\label{pemeriksaan-asumsi}}

\hypertarget{eksplorasi-kondisi-gauss-markov}{%
\subsection{Eksplorasi Kondisi
Gauss-Markov}\label{eksplorasi-kondisi-gauss-markov}}

\hypertarget{plot-sisaan-vs-y-duga}{%
\subsubsection{Plot sisaan vs Y duga}\label{plot-sisaan-vs-y-duga}}

\begin{Shaded}
\begin{Highlighting}[]
\FunctionTok{plot}\NormalTok{(model1,}\DecValTok{1}\NormalTok{) }
\end{Highlighting}
\end{Shaded}

\includegraphics{Tugas-Analisis-Regresi-Pt-7_files/figure-latex/unnamed-chunk-5-1.pdf}

Berdasarkan plot sisaan vs Y, terlihat sisaan berada di sekitar nilai
0,menunjukkan nilai harapan galat adalah nol. Namun, lebar pita untuk
setiap nilai dugaan perlu di uji lebih lanjut untuk mengetahui ragamnnya
homogen atau tidak. Bentuk pola yang didapat yaitu pola kurva, artinya
model kurva tidak pas sehingga perlu suku lain dalam model atau
transformasi terhadap Y

\hypertarget{plot-sisaan-vs-urutan}{%
\subsection{Plot sisaan vs urutan}\label{plot-sisaan-vs-urutan}}

\begin{Shaded}
\begin{Highlighting}[]
\FunctionTok{plot}\NormalTok{(}\AttributeTok{x =} \DecValTok{1}\SpecialCharTok{:}\FunctionTok{dim}\NormalTok{(data)[}\DecValTok{1}\NormalTok{],}
     \AttributeTok{y =}\NormalTok{ model1}\SpecialCharTok{$}\NormalTok{residuals,}
     \AttributeTok{type =} \StringTok{\textquotesingle{}b\textquotesingle{}}\NormalTok{, }
     \AttributeTok{ylab =} \StringTok{"Residuals"}\NormalTok{,}
     \AttributeTok{xlab =} \StringTok{"Observation"}\NormalTok{)}
\end{Highlighting}
\end{Shaded}

\includegraphics{Tugas-Analisis-Regresi-Pt-7_files/figure-latex/unnamed-chunk-6-1.pdf}

Berdasarkan plot sisaan vs urutan, terlihat tebarannya berpola, sehingga
sisaan tidak saling bebas dan model tidak pas

\hypertarget{eksplorasi-normalitas-sisaan-dengan-qq-plot}{%
\subsection{Eksplorasi Normalitas Sisaan dengan
QQ-plot}\label{eksplorasi-normalitas-sisaan-dengan-qq-plot}}

\begin{Shaded}
\begin{Highlighting}[]
\FunctionTok{plot}\NormalTok{(model1,}\DecValTok{2}\NormalTok{)}
\end{Highlighting}
\end{Shaded}

\includegraphics{Tugas-Analisis-Regresi-Pt-7_files/figure-latex/unnamed-chunk-7-1.pdf}

\hypertarget{uji-formal-kondisi-gauss-markov}{%
\section{Uji Formal Kondisi
Gauss-Markov}\label{uji-formal-kondisi-gauss-markov}}

\hypertarget{nilai-harapan-sisaan-sama-dengan-nol}{%
\subsection{1. Nilai harapan sisaan sama dengan
nol}\label{nilai-harapan-sisaan-sama-dengan-nol}}

\[
H_0 : \text{Sisaan menyebar normal}\\H_1 : \text{Sisaan tidak menyebar normal}
\]

\begin{Shaded}
\begin{Highlighting}[]
\FunctionTok{t.test}\NormalTok{(model1}\SpecialCharTok{$}\NormalTok{residuals,}\AttributeTok{mu =} \DecValTok{0}\NormalTok{,}\AttributeTok{conf.level =} \FloatTok{0.95}\NormalTok{)}
\end{Highlighting}
\end{Shaded}

\begin{verbatim}
## 
##  One Sample t-test
## 
## data:  model1$residuals
## t = -4.9493e-16, df = 14, p-value = 1
## alternative hypothesis: true mean is not equal to 0
## 95 percent confidence interval:
##  -3.143811  3.143811
## sample estimates:
##     mean of x 
## -7.254614e-16
\end{verbatim}

Diketahui bahwa p-value \textgreater{} alpha, sehingga tak tolak
\(H_0\). Oleh karena itu,dapat disimpulkan bahwa nilai harapan sisaan
sama dengan nol

\hypertarget{ragam-sisaan-homogen}{%
\subsection{2.Ragam sisaan homogen}\label{ragam-sisaan-homogen}}

\[
H_0 : \text{Ragam sisaan homogen}\\H_1 : \text{Ragam sisaan tidak homogen}
\]

\begin{Shaded}
\begin{Highlighting}[]
\FunctionTok{bptest}\NormalTok{(model1)}
\end{Highlighting}
\end{Shaded}

\begin{verbatim}
## 
##  studentized Breusch-Pagan test
## 
## data:  model1
## BP = 0.52819, df = 1, p-value = 0.4674
\end{verbatim}

Diketahui bahwa p-value \textgreater{} alpha, sehingga tak tolak
\(H_0\). Oleh karena itu,dapat disimpulkan bahwa ragam sisaan homogen

\hypertarget{sisaan-saling-bebas}{%
\subsection{3. Sisaan saling bebas}\label{sisaan-saling-bebas}}

\[
H_0 : \text{Sisaan saling bebas}\\H_1 : \text{Sisaan tidak saling bebas}
\]

\begin{Shaded}
\begin{Highlighting}[]
\FunctionTok{dwtest}\NormalTok{(model1)}
\end{Highlighting}
\end{Shaded}

\begin{verbatim}
## 
##  Durbin-Watson test
## 
## data:  model1
## DW = 0.48462, p-value = 1.333e-05
## alternative hypothesis: true autocorrelation is greater than 0
\end{verbatim}

Diketahui bahwa p-value \textless{} alpha, sehingga tolak \(H_0\). Oleh
karena itu,dapat disimpulkan bahwa sisaan tidak saling bebas

\hypertarget{uji-formal-normalitas-sisaan}{%
\section{Uji Formal Normalitas
Sisaan}\label{uji-formal-normalitas-sisaan}}

\[
H_0 : \text{Sisaan menyebar normal}\\H_1 : \text{Sisaan tidak menyebar normal}
\]

\begin{Shaded}
\begin{Highlighting}[]
\FunctionTok{shapiro.test}\NormalTok{(model1}\SpecialCharTok{$}\NormalTok{residuals)}
\end{Highlighting}
\end{Shaded}

\begin{verbatim}
## 
##  Shapiro-Wilk normality test
## 
## data:  model1$residuals
## W = 0.92457, p-value = 0.226
\end{verbatim}

Berdasarkan Shapiro-Wilk normality test diketahui bahwa p-value
\textgreater{} alpha, sehingga tak tolak \(H_0\). Oleh karena itu, dapat
disimpulkan bahwa sisaan menyebar normal.

Berdasarkan Serangkaian Uji Formal Kondisi Gauss-Markov dan Uji Formal
Normalitas Sisaan didapatkan pelanggaran asumsi Gauss-Markov yaitu,
tidak adanya autokorelasi. Karena didapatkan p-value \textless{} alpha
artinya sisaan tidak saling bebas yang seharusnya sisaan saling bebas.

\hypertarget{transformasi-data}{%
\section{Transformasi Data}\label{transformasi-data}}

\hypertarget{data-transformasi}{%
\section{Data Transformasi}\label{data-transformasi}}

\begin{Shaded}
\begin{Highlighting}[]
\NormalTok{Y }\OtherTok{=} \FunctionTok{sqrt}\NormalTok{(data}\SpecialCharTok{$}\NormalTok{Y)}
\NormalTok{X }\OtherTok{=} \FunctionTok{sqrt}\NormalTok{(data}\SpecialCharTok{$}\NormalTok{X)}
\NormalTok{data2 }\OtherTok{\textless{}{-}} \FunctionTok{data\_frame}\NormalTok{(X,Y)}
\end{Highlighting}
\end{Shaded}

\begin{verbatim}
## Warning: `data_frame()` was deprecated in tibble 1.1.0.
## i Please use `tibble()` instead.
## This warning is displayed once every 8 hours.
## Call `lifecycle::last_lifecycle_warnings()` to see where this warning was
## generated.
\end{verbatim}

\begin{Shaded}
\begin{Highlighting}[]
\NormalTok{data2}
\end{Highlighting}
\end{Shaded}

\begin{verbatim}
## # A tibble: 15 x 2
##        X     Y
##    <dbl> <dbl>
##  1  1.41  7.35
##  2  2.24  7.07
##  3  2.65  6.71
##  4  3.16  6.08
##  5  3.74  5.92
##  6  4.36  5   
##  7  5.10  4.47
##  8  5.57  4   
##  9  5.83  4.24
## 10  6.16  3.61
## 11  6.71  2.83
## 12  7.21  3.32
## 13  7.28  2.83
## 14  7.75  2   
## 15  8.06  2.45
\end{verbatim}

\hypertarget{model-regresi-1}{%
\section{Model Regresi}\label{model-regresi-1}}

\begin{Shaded}
\begin{Highlighting}[]
\NormalTok{model2 }\OtherTok{=} \FunctionTok{lm}\NormalTok{(}\AttributeTok{formula =}\NormalTok{ Y }\SpecialCharTok{\textasciitilde{}}\NormalTok{ X, data2)}
\FunctionTok{summary}\NormalTok{(model2)}
\end{Highlighting}
\end{Shaded}

\begin{verbatim}
## 
## Call:
## lm(formula = Y ~ X, data = data2)
## 
## Residuals:
##      Min       1Q   Median       3Q      Max 
## -0.42765 -0.17534 -0.05753  0.21223  0.46960 
## 
## Coefficients:
##             Estimate Std. Error t value Pr(>|t|)    
## (Intercept)  8.71245    0.19101   45.61 9.83e-16 ***
## X           -0.81339    0.03445  -23.61 4.64e-12 ***
## ---
## Signif. codes:  0 '***' 0.001 '**' 0.01 '*' 0.05 '.' 0.1 ' ' 1
## 
## Residual standard error: 0.2743 on 13 degrees of freedom
## Multiple R-squared:  0.9772, Adjusted R-squared:  0.9755 
## F-statistic: 557.3 on 1 and 13 DF,  p-value: 4.643e-12
\end{verbatim}

\begin{Shaded}
\begin{Highlighting}[]
\NormalTok{model2}
\end{Highlighting}
\end{Shaded}

\begin{verbatim}
## 
## Call:
## lm(formula = Y ~ X, data = data2)
## 
## Coefficients:
## (Intercept)            X  
##      8.7125      -0.8134
\end{verbatim}

Didapatkan Model Regresi sebagai berikut: \[\hat Y = 8.7125-0.8134X\]

\#Eksplorasi \#\# Plot Hubungan X dan Y Asumsi bentuk model regresi
linear sederhana dengan plot antara X dan Y

\begin{Shaded}
\begin{Highlighting}[]
\FunctionTok{plot}\NormalTok{(}\AttributeTok{x=}\NormalTok{data2}\SpecialCharTok{$}\NormalTok{X,}\AttributeTok{y=}\NormalTok{data2}\SpecialCharTok{$}\NormalTok{Y)}
\end{Highlighting}
\end{Shaded}

\includegraphics{Tugas-Analisis-Regresi-Pt-7_files/figure-latex/unnamed-chunk-14-1.pdf}

\hypertarget{pemeriksaan-asumsi-1}{%
\section{Pemeriksaan Asumsi}\label{pemeriksaan-asumsi-1}}

\hypertarget{eksplorasi-kondisi-gauss-markov-1}{%
\subsection{Eksplorasi Kondisi
Gauss-Markov}\label{eksplorasi-kondisi-gauss-markov-1}}

\hypertarget{plot-sisaan-vs-y-duga-1}{%
\subsubsection{Plot sisaan vs Y duga}\label{plot-sisaan-vs-y-duga-1}}

\begin{Shaded}
\begin{Highlighting}[]
\FunctionTok{plot}\NormalTok{(model2,}\DecValTok{1}\NormalTok{) }
\end{Highlighting}
\end{Shaded}

\includegraphics{Tugas-Analisis-Regresi-Pt-7_files/figure-latex/unnamed-chunk-15-1.pdf}

\hypertarget{plot-sisaan-vs-urutan-1}{%
\subsection{Plot sisaan vs urutan}\label{plot-sisaan-vs-urutan-1}}

\begin{Shaded}
\begin{Highlighting}[]
\FunctionTok{plot}\NormalTok{(}\AttributeTok{x =} \DecValTok{1}\SpecialCharTok{:}\FunctionTok{dim}\NormalTok{(data2)[}\DecValTok{1}\NormalTok{],}
     \AttributeTok{y =}\NormalTok{ model2}\SpecialCharTok{$}\NormalTok{residuals,}
     \AttributeTok{type =} \StringTok{\textquotesingle{}b\textquotesingle{}}\NormalTok{, }
     \AttributeTok{ylab =} \StringTok{"Residuals"}\NormalTok{,}
     \AttributeTok{xlab =} \StringTok{"Observation"}\NormalTok{)}
\end{Highlighting}
\end{Shaded}

\includegraphics{Tugas-Analisis-Regresi-Pt-7_files/figure-latex/unnamed-chunk-16-1.pdf}

\hypertarget{eksplorasi-normalitas-sisaan-dengan-qq-plot-1}{%
\subsection{Eksplorasi Normalitas Sisaan dengan
QQ-plot}\label{eksplorasi-normalitas-sisaan-dengan-qq-plot-1}}

\begin{Shaded}
\begin{Highlighting}[]
\FunctionTok{plot}\NormalTok{(model2,}\DecValTok{2}\NormalTok{)}
\end{Highlighting}
\end{Shaded}

\includegraphics{Tugas-Analisis-Regresi-Pt-7_files/figure-latex/unnamed-chunk-17-1.pdf}

\hypertarget{uji-formal-kondisi-gauss-markov-1}{%
\section{Uji Formal Kondisi
Gauss-Markov}\label{uji-formal-kondisi-gauss-markov-1}}

\hypertarget{nilai-harapan-sisaan-sama-dengan-nol-1}{%
\subsection{1. Nilai harapan sisaan sama dengan
nol}\label{nilai-harapan-sisaan-sama-dengan-nol-1}}

\[
H_0 : \text{Nilai harapan sisaan sama dengan 0}\\H_1 : \text{Nilai harapan tidak sama dengan 0}
\]

\begin{Shaded}
\begin{Highlighting}[]
\FunctionTok{t.test}\NormalTok{(model2}\SpecialCharTok{$}\NormalTok{residuals,}\AttributeTok{mu =} \DecValTok{0}\NormalTok{,}\AttributeTok{conf.level =} \FloatTok{0.95}\NormalTok{)}
\end{Highlighting}
\end{Shaded}

\begin{verbatim}
## 
##  One Sample t-test
## 
## data:  model2$residuals
## t = 2.0334e-16, df = 14, p-value = 1
## alternative hypothesis: true mean is not equal to 0
## 95 percent confidence interval:
##  -0.1463783  0.1463783
## sample estimates:
##    mean of x 
## 1.387779e-17
\end{verbatim}

Diketahui bahwa p-value \textgreater{} alpha, sehingga tak tolak
\(H_0\). Oleh karena itu,dapat disimpulkan bahwa nilai harapan sisaan
sama dengan nol

\hypertarget{ragam-sisaan-homogen-1}{%
\subsection{2.Ragam sisaan homogen}\label{ragam-sisaan-homogen-1}}

\[
H_0 : \text{Ragam sisaan homogen}\\H_1 : \text{Ragam sisaan tidak homogen}
\]

\begin{Shaded}
\begin{Highlighting}[]
\FunctionTok{ncvTest}\NormalTok{(model2)}
\end{Highlighting}
\end{Shaded}

\begin{verbatim}
## Non-constant Variance Score Test 
## Variance formula: ~ fitted.values 
## Chisquare = 2.160411, Df = 1, p = 0.14161
\end{verbatim}

Diketahui bahwa p-value \textgreater{} alpha, sehingga tak tolak
\(H_0\). Oleh karena itu,dapat disimpulkan bahwa ragam sisaan homogen

\hypertarget{sisaan-saling-bebas-1}{%
\subsection{3. Sisaan saling bebas}\label{sisaan-saling-bebas-1}}

\[
H_0 : \text{Sisaan saling bebas}\\H_1 : \text{Sisaan tidak saling bebas}
\]

\begin{Shaded}
\begin{Highlighting}[]
\FunctionTok{dwtest}\NormalTok{(model2)}
\end{Highlighting}
\end{Shaded}

\begin{verbatim}
## 
##  Durbin-Watson test
## 
## data:  model2
## DW = 2.6803, p-value = 0.8629
## alternative hypothesis: true autocorrelation is greater than 0
\end{verbatim}

Diketahui bahwa p-value \textgreater{} alpha, sehingga tak tolak
\(H_0\). Oleh karena itu,dapat disimpulkan bahwa sisaan saling bebas

\hypertarget{uji-formal-normalitas-sisaan-1}{%
\section{Uji Formal Normalitas
Sisaan}\label{uji-formal-normalitas-sisaan-1}}

\[ H_0 : \text{Sisaan menyebar normal}\\H_1 : \text{Sisaan tidak menyebar normal} \]

\begin{Shaded}
\begin{Highlighting}[]
\FunctionTok{shapiro.test}\NormalTok{(model2}\SpecialCharTok{$}\NormalTok{residuals)}
\end{Highlighting}
\end{Shaded}

\begin{verbatim}
## 
##  Shapiro-Wilk normality test
## 
## data:  model2$residuals
## W = 0.96504, p-value = 0.7791
\end{verbatim}

Berdasarkan Shapiro-Wilk normality test diketahui bahwa p-value
\textgreater{} alpha, sehingga tak tolak \(H_0\). Oleh karena itu, dapat
disimpulkan bahwa sisaan menyebar normal.

Berdasarkan transformasi yang telah dilakukan didapat model regresi yang
lebih baik dan efektif disertai dengan semua asumsi Gauss Markov dan
Normalitas sudah terpenuhi dalam analisis regresi linear sederhana.
Sehingga model regresi terbaik dari data ini adalah:
\[\hat Y = 8.7125-0.8134X\]

\end{document}
